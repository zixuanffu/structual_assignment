\documentclass[10pt]{article}
\usepackage{fullpage,graphicx,psfrag,amsmath,amsfonts,verbatim}
\usepackage[small,bf]{caption}
\usepackage{amsthm}
% \usepackage[hidelinks]{hyperref}
\usepackage{hyperref}
\usepackage{bbm} % for the indicator function to look good
\usepackage{color}
\usepackage{mathtools}
\usepackage{fancyhdr} % for the header
\usepackage{booktabs} % for regression table display (toprule, midrule, bottomrule)
\usepackage{adjustbox} % for regression table display
\usepackage{threeparttable} % to use table notes
\usepackage{natbib} % for bibliography
\usepackage{tikz}
\usetikzlibrary{arrows.meta}
\input newcommand.tex
\bibliographystyle{apalike}
% \setlength{\parindent}{0pt} % remove the automatic indentation

\title{A dynamic model of personality, schooling, and occupational choice (ToddZhang2020QE)}
\author{Fu Zixuan}
\date{\today}

\begin{document}
\maketitle
% \thispagestyle{empty}
% \begin{abstract}

% \end{abstract}

% \newpage
% \thispagestyle{empty}
% \tableofcontents
% \newpage

% \setcounter{page}{1}

\section{Summary}
The paper by \citet{todd2020dynamic} focuses on the individual's choice of schooling/working at each period as the optimal solution to maximizing lifetime utility. This falls under the classical single-agent dynamic problem with finite periods. The dynamic decision at each period is the individual's education and occupation choice. The state that evolves over time includess an individual's schooling/working experienc,e which grows over time, the cognitive abilit,y which is tim- invarian,t as well as the novel addition, the individual personality measured infive5 dimensiosns, which also develops over time. While the state is (noiisily) observed, the model adds an unobserved element -- latent type, which interplays with the observed state and the utility functions. By establishing a rich model that incorporates cognitive abilitive personality tr,aits and latent type, the authaimant to answer the question of how personality evolves over time in response to age and experience. Secondly, though the model itself has already restricted the chansnel through which one variale affescts another, the estimation of model parameters hselps quantify the magnitude of impact. For example, the latent type transitions to the next period type depending on the current type and the current personality. One can  for whether latent type is time-variant using the parameter estimates.The paper by \citet{todd2020dynamic} focuses on the individual's choice of schooling/working at each lsriod as the optimal solution to maximizing lifetime utility. This falls under the cl,assical single agent dynamic problem with finite periods. The dynamic decision at seach period is the individual's education and occupation choice. The state t thehat evolves over time include an ind theividual's schooli theng/working experience which grows over time, the cognitivility which is time iiesvariant as well as the novel addition, the individualepersonality measured in 5 dimension, which also devel,ops over time. While the state s (noiarey) observed, the model adds an unobserved element --s latent type, which interplays with the observed state and the utility fnctions. By  a rich model that incorporates cognitisve ability, incognitve personality traits and latent type, the authors wants to answer the question of how personality evolves over time in response to age and experience. nSeconof though the model itself has already restricted the channel through which one variathe es affect another, the estimation of model parameters help quantify the magnitude of impact. For example, the latent type transitions to the next perio- type dep,ending on-the current type andthe current personality. One can test for whether latent type is time-variant using the parameter estimates. Lastly, the model allows the simulation of two different education policies (tuition reduction and compuslory schooling). While most of the identification results rely on previous literature notably that of \cite{hu2012nonparametric}, the paper imposes further restriction as in \citet{hu2017simple} to identify the probability distribution of CCP, law of motion of latent type k, and law of motion of observed state. Once the nonparametric identification of the the above probability is established, the paper resorts to the simulated mathods of moments to estimate all the model parameters though the identification arguements is hard to establish when using moment-based estimator. The next section will present the structural model in a simplified way, and discuss a bit about the estimation and simulation exercise. Afterwards, I will try to link the identification arguments in the paper to what we learned in class. While it seems to me that in the paper, the identification is indepedent from estimation, I hope to establish a connection between the two, and how one can aid another. Lastly, to emphasize the role of unobserved heterogeneity, I will summarize cases where there is no unobserved heterogeneity (latent type), time invariant and time variant latent type.
\section{Model}

\section{Discussion}

\pagebreak \newpage \bibliography{../References/ref.bib}

\end{document}

